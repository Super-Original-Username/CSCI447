\documentclass[twoside,11pt]{article}
% Any additional packages needed should be included after jmlr2e.
% Note that jmlr2e.sty includes epsfig, amssymb, natbib and graphicx,
% and defines many common macros, such as 'proof' and 'example'.
%
% It also sets the bibliographystyle to plainnat; for more information on
% natbib citation styles, see the natbib documentation, a copy of which
% is archived at http://www.jmlr.org/format/natbib.pdf

\usepackage{jmlr2e}

% Definitions of handy macros can go here

\newcommand{\dataset}{{\cal D}}
\newcommand{\fracpartial}[2]{\frac{\partial #1}{\partial  #2}}

% Heading arguments are {volume}{year}{pages}{date submitted}{date published}{paper id}{author-full-names}

\jmlrheading{1}{2019}{1-7}{9/19}{9/19}{meila00a}{Ethan Fison, Zan Monteith, Alex Salois, Sage Acteson}

% Short headings should be running head and authors last names

\ShortHeadings{Simple Validation of the Naïve Bayes Algorithm}{Fison, Monteith, Salois, Acteson}
\firstpageno{1}

\begin{document}

\title{Simple Validation of the Naïve Bayes Algorithm}

\author{\name Ethan Fison \email ethanfison@gmail.com \\
        \name Zan Monteith \email email \\
        \name Alex Salois \email email \\ 
        \name Sage Acteson \email email \\ 
       \addr Gianforte School of Computing\\
       Montana State University\\
       Bozeman, MT 59715, USA}


\editor{Ethan, Zan, Alex, Sage}

\maketitle

\begin{abstract}%   <- trailing '%' for backward compatibility of .sty file
This paper describes the Na{\"i}ve Bayes algorithm, used for classification
of data. This algorithm builds its model by finding the average value for each
attribute of a given class, then classifies an input by finding the class it 
most closely matches. In this experiement, we run ten-fold cross-validation 
of our models built using 5 different datasets acquired from the UCI Machine 
Learning Repositoryto test the accuracy of the algorithm\\
(this needs to also briefly describe our results)

\end{abstract}

\begin{keywords}
  Na{\"i}ve Bayes, Classification, Validation
\end{keywords}

\section{Introduction}

The goal of this experiment is to validate the efficacy of the Naïve Bayes 
algorithm. given 5 predetermined datasets from the UCI Machine Learning Repository.
To do so, we will employ ten-fold cross-validation on each set, with the order of 
entries in the data scrambled prior to being split and passed through Naïve Bayes.
To further examine the capabiities of Naïve Bayes, we will then pass a modified 
version of the original data, this time with the attributes of 10\% (rounded up) of 
classes scrambled within their entries. For this experiment, we have kept our 
statistical analysis simple, building a confusion matrix with the results of each run,
and using that to calculate our error, precision, and recall values for the model 
generated in that run.\\


\section{Hypotheses}

For each set, we assume the null hypothesis. That is, we assume that there will be no 
significant change between the original values and reclassified values for any given test
set sent into our model.\\

When scrambling data, we expect there to be a measurable increase in error for each classification.

\emph{this needs better wording. pls help}\\
\section{Implementation Decisions}

This section will discuss the design choices made in the development of each major 
python file in the project, in addition to a description of the functionality of 
the file

\subsection{File Input}

\subsection{Ten-fold Cross-Validation}

The implementation of the ten-fold cross-validation was intended to be straightforward. 
Due to the nature of this function involving calls to each other python file in our 
project, it was determined that the TenFCV python file would also serve as our driver. 
There are several functions performed within this file. In addition to performing 
ten-fold cross-validation of our It supports functionalityfor scrambling values for 
10\%(rounded up) of rows within a given dataset, scrambling of the overall order of 
rows within the dataset.\\
The first pass of this function for any given dataset will scramble the order of rows, 
split the datasetinto ten (mostly) equally-sized arrays, then sequentially feed 9 arrays 
into the learner function within the algorithm, followed by feeding the remaining array, 
now with class attributes removed, into the classifier function. The results given by 
the classifer are then fed into our statistical analysis functions along with the original 
segment of the reclassified data for comparison.

\subsection{Naïve Bayes Implementation}

\subsection{Statistical Analysis}


\section{Results}

\emph{this could use things like data from a sample run, i.e. a confusion matrix, the error/precision/recall, etc.}


% Acknowledgements should go at the end, before appendices and references

\acks{\emph{we should probably put something. Maybe thank Dr. Sheppard for offering this class or something? I dunno}}

% Manual newpage inserted to improve layout of sample file - not
% needed in general before appendices/bibliography.

\newpage

\appendix
\section*{Appendix A.}
\label{app:theorem}

% Note: in this sample, the section number is hard-coded in. Following
% proper LaTeX conventions, it should properly be coded as a reference:

%In this appendix we prove the following theorem from
%Section~\ref{sec:textree-generalization}:

In this appendix we prove the following theorem from
Section~6.2:

\noindent
{\bf Theorem} {\it Let $u,v,w$ be discrete variables such that $v, w$ do
not co-occur with $u$ (i.e., $u\neq0\;\Rightarrow \;v=w=0$ in a given
dataset $\dataset$). Let $N_{v0},N_{w0}$ be the number of data points for
which $v=0, w=0$ respectively, and let $I_{uv},I_{uw}$ be the
respective empirical mutual information values based on the sample
$\dataset$. Then
\[
	N_{v0} \;>\; N_{w0}\;\;\Rightarrow\;\;I_{uv} \;\leq\;I_{uw}
\]
with equality only if $u$ is identically 0.} \hfill\BlackBox

\noindent
{\bf Proof}. We use the notation:
\[
P_v(i) \;=\;\frac{N_v^i}{N},\;\;\;i \neq 0;\;\;\;
P_{v0}\;\equiv\;P_v(0)\; = \;1 - \sum_{i\neq 0}P_v(i).
\]
These values represent the (empirical) probabilities of $v$
taking value $i\neq 0$ and 0 respectively.  Entropies will be denoted
by $H$. We aim to show that $\fracpartial{I_{uv}}{P_{v0}} < 0$....\\

{\noindent \em Remainder omitted in this sample. See http://www.jmlr.org/papers/ for full paper.}




\vskip 0.2in
\bibliography{sample}

\end{document}
